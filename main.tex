\documentclass{amsart}

\usepackage{amssymb}

\usepackage{amsthm}
\theoremstyle{definition}
\newtheorem{definition}{Definition}
\newtheorem{theorem}{Theorem}
\newtheorem{corollary}{Corollary}

\usepackage{parskip}
\makeatletter
\def\thm@space@setup{\thm@preskip=\parskip\thm@postskip=0pt}
\makeatother

\title{Galois Parsimony}
\author{Gregory Lim}
\date{\today}

\begin{document}

\begin{abstract}
	We prove and interpret the Galois connection induced by satisfaction between sets of sentences and classes of structures in first-order logic.
\end{abstract}

\maketitle

Let $\mathcal{L}$ be a first-order language,
and $\mathcal{S}$ be a set of $\mathcal{L}$-sentences,
and $\mathcal{M}$ be a class of $\mathcal{L}$-structures.

\begin{definition}
	Let $\Phi\subseteq\mathcal{S}$.
	\[
		\operatorname{Mod}(\Phi)
		=
		\{
			m\in\mathcal{M}
			\mid
			\forall\varphi\in\Phi\,(m\vDash\varphi)
		\}
	\]
\end{definition}

\begin{definition}
	Let $M\subseteq\mathcal{M}$.
	\[
		\operatorname{Th}(M)
		=
		\{
			\varphi\in\mathcal{S}
			\mid
			\forall m\in M\,(m\vDash\varphi)
		\}
	\]
\end{definition}

\begin{theorem}
	Let $\Phi\subseteq\mathcal{S}$, and $M\subseteq\mathcal{M}$.
	\[
		\Phi\subseteq\operatorname{Th}(M)
		\iff
		M\subseteq\operatorname{Mod}(\Phi)
	\]
\end{theorem}

\begin{proof}
	\[
		\Phi\subseteq\operatorname{Th}(M)
		\iff
		\forall\varphi\in\Phi,\,\forall m\in M\,(m\vDash\varphi)
		\iff
		M\subseteq\operatorname{Mod}(\Phi)
	\]
\end{proof}

The syntax--semantics correspondence reverses inclusion: more axioms yield fewer models.
This is a precise sense in which more constraints yield fewer possible worlds.

\end{document}
